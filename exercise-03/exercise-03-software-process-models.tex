% Document class, paper size, base font size
\documentclass[a4paper, 12pt]{article}

% Encoding
\usepackage[utf8]{inputenc}
\usepackage[T1]{fontenc}

% Prevent indentation of beginning of paragraphs
\setlength{\parindent}{0em}

% Space between paragraphs
\parskip 0.5em

% Title, authors and date
\title{Exercise 3: Software Process Models}
\author{
  Kendra Birringer (1229372)\and
  Nader Cacace (1208115)\and
  Steffen Hanzlik (1207417)\and
  Marco Peluso (1228849)\and
  Svetozar Stojanovic (1262287)\and
}
\date{November 9th 2019}

% Begin actual document
\begin{document}

\maketitle
\newpage

\section{Case 1}
\subsection{a)}
Another company built the existing software project, so we have no idea of the existing software project. Since the other software company doesn't exist anymore, so we are not able to contact them.

Requirements change very often. It is questionable, if we know, what the customer really wants.

\subsection{b)}
Try to search for similar projects and compare the "work load". Speak with the customer and try to get all information that we need to work on the project. With these informations we try to estimate the best possible price. 

Maybe see the other software to get a feeling for our task.

Cheap starting price and for additional features the price will increase.

\subsection{c)}
Incremental, because we can start with less analysis of the customer requirements at the beginning (it's possible, that we don't have all informations).

We can work very close with the customer and if the customer changed his idea, we can adapt pretty fast to the new requirements.

We can deliver usable software, quickly.

Frequent customer feedback.

\newpage
\section{Case 2}
\subsection{Ideas}
\begin{itemize}
    \item Scrum
    \item Lean Design
\end{itemize}

\subsection{Decision}
The main concept is, we are two (CS?) students with an idea for a mobile app, which should become really successful.

We decided against \emph{Scrum}, because there we need a large development team, which actually does not exist, because we have a really small team.

Therefore, we concluded that the best solution for our project is to use an agile software development process called \emph{Lean Development Process}, because we are focused on customer interaction and the whole process is inexpensive and is good for small development teams.

\subsection{Final words}
For our case the Lean Development Process fits in best way to our scenario. But we came to the point, that every agile development process would fit to our case.

\newpage
\section{Case 3}
\subsection{a)}
According to the needs of the customer and the lively competition, we are obliged to be very efficient. Another factor is that the customer has a list of requirements in form of a list, that will most likely have to be changed in the future.

We decided to use an agile software process. The customer will most likely make changes to their original list of requirements during development, so we have to be able to adopt to those changes. An agile software process will allow us to be flexible enough to react to changes to the requirements.

A problem is that we are not able to make a perfect work schedule. The object-oriented design will allow us to change the already existing software more efficient.

\subsection{b)}
First we will ask the customer for the list of the additional software functionality, so we can get an idea of the scope of the project.

Then we will approach the customer to talk with them about their requirement list. After that, we will get the necessary knowledge about their software. The next step is to make a price offer and make a contract with the customer. Then we will give them feedback and will continue with feedback through the work process.

We will create short sprints, in which we will implement a few features and will show the customer the software after each increment, get additional feedback and update the requirements list according to that feedback.

\end{document}
