% Document class, paper size, base font size
\documentclass[a4paper, 12pt]{article}

% Encoding
\usepackage[utf8]{inputenc}
\usepackage[T1]{fontenc}

% Prevent indentation of beginning of paragraphs
\setlength{\parindent}{0em}

% Space between paragraphs
\parskip 0.5em

% include PDFs
%\usepackage{pdfpages}

% For Check-List
\usepackage{enumitem}
\usepackage{amssymb,wasysym}
\usepackage{graphicx}
\usepackage{enumerate}
\newlist{todolist}{itemize}{2}
\setlist[todolist]{label=$\square$}
% Title, authors and date
\title{E-Scooter Project 01: Definition of Done}
\author{
    Kendra Birringer (1229372)\\
    Nader Cacace (1208115)\\
    Steffen Hanzlik (1207417)\\
    Marco Peluso (1228849)\\
    Svetozar Stojanovic (1262287)\\
    \\
    Frankfurt University of Applied Sciences
}
\date{December 14th, 2019}


% Begin actual document
\begin{document}

\maketitle
\newpage
\tableofcontents

\newpage
\section{Introduction}
In this project, we will take the role of founding a start-up company in the E-Scooter rental business.
For analyzing this project, our team decided to use the agile SCRUM method.

The purpose of this project is a clear understanding of requirements, developing use case scenarios and UML diagrams for describing the planned software product.
Furthermore, the objective is to develop a high-level software architecture and build user interface prototypes for relevant parts of the functionality.
For our project, the artifacts consist of diagrams and documentation of for describing the software functionality for outsourcing the development. The quality of these artifacts will be ensured by a proper definition of "Done".

The first step of our project is to collect additional requirements and refine these to build a list of backlog items and estimating the time for producing the artifacts for each backlog item. The purpose of a backlog item is to take a requirement description, in form of a use case description, as input and build an analysis model of our software capability as result.

\section{Definition of "Done"}
The definition of "Done" (DoD) is part of the SCRUM metrics. All members of the Scrum Team must have a shared understanding of what it means when the work is complete, to ensure transparency. \cite{scrumguide}
It is a (check-)list of items which need to be validated to consider a backlog item being “Done”. DoD is defined by the development organization to make sure that the results of multiple teams can be integrated into a releasable product  \cite{thoma1}

For our project, the result needs to match the following definition of "Done:
\begin{todolist}

\item Description of the requirement in form of a Use Case
\item Categorization of requirement (functional/non-functional, client/ server)
\item Business value of the corresponding functionality
\item Effort estimation for the implementation of the requirement
\item For UI related functions: UI prototype
\item UML Diagrams 

    $\bullet$ Use case diagram
    
    $\bullet$ Activity diagram
    
    $\bullet$ Class diagram
    
    $\bullet$ Sequence diagram
    
    Depending on the type of the requirement, not all these diagrams may be applicable. If a diagram is omitted, describe why.
\item Detailed documentation (e.g. table) about who worked on the item and what has been done during the sprint.
\item Overall quality of the documentation meets general industry standards.
\item The results have been reviewed and accepted by another member of the team (tester). It needs to be documented who has performed the review.

\end{todolist}



\newpage    

\begin{thebibliography}{}
\bibitem{scrumguide}
https://www.scrumguides.org/scrum-guide.html\#artifact-transparency-done 

\bibitem{thoma1}
 Prof. Dr.-Ing Peter Thoma \emph{02-3 Software Engineering Analysis (Scrum)}

\end{thebibliography}

%\listoffigures
\end{document}
