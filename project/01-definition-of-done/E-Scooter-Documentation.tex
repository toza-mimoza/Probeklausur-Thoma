% Document class, paper size, base font size
\documentclass[a4paper, 12pt]{article}

% Encoding
\usepackage[utf8]{inputenc}
\usepackage[T1]{fontenc}

% Prevent indentation of beginning of paragraphs
\setlength{\parindent}{0em}

% Space between paragraphs
\parskip 0.5em

% include PDFs
%\usepackage{pdfpages}

% For Check-List
\usepackage{enumitem}
\usepackage{amssymb,wasysym}
\usepackage{graphicx}
\usepackage{enumerate}
\newlist{todolist}{itemize}{2}
\setlist[todolist]{label=$\square$}

% Title, authors and date
\title{Software Engineering - Analysis\\
Project: E-Scooter Rental Service}

\author{
    Kendra Birringer (1229372)\\
    Nader Cacace (1208115)\\
    Steffen Hanzlik (1207417)\\
    Marco Peluso (1228849)\\
    Svetozar Stojanovic (1262287)\\
    \\
    Frankfurt University of Applied Sciences
}
\date{February 15th, 2020}


% Begin actual document
\begin{document}

\maketitle
\newpage
\tableofcontents

\newpage
\section{Introduction}
In this project, we will take the role of founding a start-up company in the E-Scooter rental business.
For analyzing this project, our team decided to use the agile SCRUM method.

The purpose of this project is a clear understanding of requirements, developing use case scenarios and UML diagrams for describing the planned software product.
Furthermore, the objective is to develop a high-level software architecture and build user interface prototypes for relevant parts of the functionality.
For our project, the artifacts consist of diagrams and documentation of for describing the software functionality for outsourcing the development. The quality of these artifacts will be ensured by a proper definition of "Done".

The first step of our project is to collect additional requirements and refine these to build a list of backlog items and estimating the time for producing the artifacts for each backlog item. The purpose of a backlog item is to take a requirement description, in form of a use case description, as input and build an analysis model of our software capability as result.

\section{Definition of "Done"}
The definition of "Done" (DoD) is part of the SCRUM metrics. All members of the Scrum Team must have a shared understanding of what it means when the work is complete, to ensure transparency. \cite{scrumguide}
It is a (check-)list of items which need to be validated to consider a backlog item being “Done”. DoD is defined by the development organization to make sure that the results of multiple teams can be integrated into a releasable product.  \cite{thoma1}

For our project, the result needs to match the following definition of "Done":
\begin{todolist}

\item Description of the requirement in form of a Use Case
\item Categorization of requirement (functional/non-functional, client/ server)
\item Business value of the corresponding functionality
\item Effort estimation for the implementation of the requirement
\item For UI related functions: UI prototype
\item UML Diagrams 

    $\bullet$ Use case diagram
    
    $\bullet$ Activity diagram
    
    $\bullet$ Class diagram
    
    $\bullet$ Sequence diagram
    
    Depending on the type of the requirement, not all these diagrams may be applicable. If a diagram is omitted, describe why.
\item Detailed documentation (e.g. table) about who worked on the item and what has been done during the sprint.
\item Overall quality of the documentation meets general industry standards.
\item The results have been reviewed and accepted by another member of the team (tester). It needs to be documented who has performed the review.

\end{todolist}


%==============================================================================
\section{Backlog Items}

Insert pdf-Document here!
%==============================================================================

%==============================================================================
\section{Week One}
%==============================================================================
\begin{itemize}
\item start to collect Requirements for the E-Scooter Project
\item discussion which Requirement can fit
\item save every Requirement in a google excel sheet
\item talk about estimation, satisfactions, dissatifactions, priority, what the requirement should do, which classes we need for it and when the requirement fit
\item scrum meetings over Discord
\item every teammember collect more Requirement at home 
\item think about how these E-Scooter should or can work and interact with the client
\item test a ride with an E-Scooter from Uber do get a imagination how the main concept is.

\end{itemize}

%==============================================================================
\section{Week Two}
%==============================================================================
\begin{itemize}
\item discussions about the setted Requirements, which fits which is to much or not necessary. 
\item collect more Requirements at least 40 up to 50 
\item start to talk about Use Case Diagramms, Use Cases and Class Diagram
\item build a first concept of an Use Case Diagramm
\item first Use Case Diagram was really big and must be adjusted in the future
\item Build a Definition of done 
\item build a main structure for documentation
\end{itemize}
%==============================================================================
\section{Week Three}
%===============================================================================
\begin{itemize}
\item modified Use Case Diagram 
\item model some Activity and Sequence Diagrams 
\item First Prototype of the UI
\item talk about actual use cases and modyfing or adjust it to get smaller
\item think about an Class Diagram
\item which classe do we need ?
\item which specification do we have 
\item which multiplicity do we have ?
\item start with documentation over every Requirement
\item think about where do we need Sequence Diagrams 
\item think about where do we need Activity Daigrams
	\begin{itemize}
		\item the Use Case we do not implemented a Sequence Diagram we note here whith an explanation why.(e.g We do not build a Sequence Diagram for Give an Feedback, because it is to laim :) ) 
		\item The same for Activity Diagrams
	\end{itemize}
\item start the Class Diagram 
\item modify and adjust our UI Prototype
\item finished the Use Case Diagram
\end{itemize}
%==============================================================================
\section{Week Four}
%===============================================================================
\begin{itemize}
\item Week for adjust and modification
\item What is missing ? 
\item Adjusting a last time the Requirements 
\item on Base of the final Requirements
\item we finished the UI Prototype 
\item we finished the Class Diagrams
\item we adjusted the Activity Diagrams(no cancelation)
\item talk about the Presentation in Februrary
\item adjusted the Sequence Diagrams
\item talk about some missbuilded Diagrams and modyfied them
\item talk about Documentation in the Definition of done
\item adjust the documetation for the Use Cases 
\end{itemize}
%==============================================================================
\section{Week Five}
%===============================================================================
\begin{itemize}
\item start to build the Presentation
\item discussion about which role every member get in this project and how he could integrate for this project
\item finished the activity diagram
\item finished the sequence diagram
\item finished the documentation
\item finished the presentation
\item generate Magic Draw report and add it to the Documentation
\item add the scrum protocols ot the Documentation
\item talk about the workflow table who did what and add it to the documentation
\item add References to Documentation 
\item finished the project at this moment
\item think about further work, which requirement we would build in at this moment we not implemented it
\item think about the design engeneering 
\item which functions we would build in the next time
\end{itemize}
%==============================================================================

\newpage    
%==============================================================================
\begin{thebibliography}{}
\bibitem{scrumguide}
https://www.scrumguides.org/scrum-guide.html\#artifact-transparency-done 

\bibitem{thoma1}
 Prof. Dr.-Ing Peter Thoma \emph{02-3 Software Engineering Analysis (Scrum)}

\end{thebibliography}

%\listoffigures

\newpage
%==============================================================================
\section{Appendix}
%==============================================================================
\subsection{Workflow Table Week One}
\begin{table}[h]
\centering
\setlength{\tabcolsep}{12pt}
\begin{tabular}{|c|c|c|c|c|}
\hline
K. Birringer & N. Cacace & S. Hanzlik & M. Peluso & S. Stojanovic\\
\hline
\% & \% & \% & \% & \% \\ 
\hline
\end{tabular}
\end{table}
%==============================================================================
\subsection{Workflow Table Week Two}
\begin{table}[h]
\centering
\setlength{\tabcolsep}{12pt}
\begin{tabular}{|c|c|c|c|c|}
\hline
K. Birringer & N. Cacace & S. Hanzlik & M. Peluso & S. Stojanovic\\
\hline
\% & \% & \% & \% & \% \\ 
\hline
\end{tabular}
\end{table}
%==============================================================================
\subsection{Workflow Table Week Three}

\begin{table}[h]
\centering
\setlength{\tabcolsep}{12pt}
\begin{tabular}{|c|c|c|c|c|}
\hline
K. Birringer & N. Cacace & S. Hanzlik & M. Peluso & S. Stojanovic\\
\hline
\% & \% & \% & \% & \% \\ 
\hline
\end{tabular}
\end{table}
\newpage
%==============================================================================
\subsection{Workflow Table Week Four}

\begin{table}[h]
\centering
\setlength{\tabcolsep}{12pt}
\begin{tabular}{|c|c|c|c|c|}
\hline
K. Birringer & N. Cacace & S. Hanzlik & M. Peluso & S. Stojanovic\\
\hline
\% & \% & \% & \% & \% \\ 
\hline
\end{tabular}
\end{table}
%==============================================================================
\subsection{Workflow Table Week Five}

\begin{table}[h]
\centering
\setlength{\tabcolsep}{12pt}
\begin{tabular}{|c|c|c|c|c|}
\hline
K. Birringer & N. Cacace & S. Hanzlik & M. Peluso & S. Stojanovic\\
\hline
\% & \% & \% & \% & \% \\ 
\hline
\end{tabular}
\end{table}
%==============================================================================
\subsection{Meeting Protocols}


\end{document}
