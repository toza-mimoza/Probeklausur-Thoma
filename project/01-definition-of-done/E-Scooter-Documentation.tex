% Document class, paper size, base font size
\documentclass[a4paper, 12pt]{article}

% Encoding
\usepackage[utf8]{inputenc}
\usepackage[T1]{fontenc}

% Prevent indentation of beginning of paragraphs
\setlength{\parindent}{0em}

% Space between paragraphs
\parskip 0.5em

% include PDFs
%\usepackage{pdfpages}

% For Check-List
\usepackage{enumitem}
\usepackage{amssymb,wasysym}
\usepackage{graphicx}
\usepackage{enumerate}
\newlist{todolist}{itemize}{2}
\setlist[todolist]{label=$\square$}

% Title, authors and date
\title{Software Engineering - Analysis\\
Project: E-Scooter Rental Service}

\author{
    Kendra Birringer (1229372)\\
    Nader Cacace (1208115)\\
    Steffen Hanzlik (1207417)\\
    Marco Peluso (1228849)\\
    Svetozar Stojanovic (1262287)\\
    \\
    Frankfurt University of Applied Sciences
}
\date{February 15th, 2020}


% Begin actual document
\begin{document}

\maketitle
\newpage
\tableofcontents

\newpage
\section{Introduction}
In this project, we will take the role of founding a start-up company in the E-Scooter rental business.
For analyzing this project, our team decided to use the agile SCRUM method.

The purpose of this project is a clear understanding of requirements, developing use case scenarios and UML diagrams for describing the planned software product.
Furthermore, the objective is to develop a high-level software architecture and build user interface prototypes for relevant parts of the functionality.
For our project, the artifacts consist of diagrams and documentation of for describing the software functionality for outsourcing the development. The quality of these artifacts will be ensured by a proper definition of "Done".

The first step of our project is to collect additional requirements and refine these to build a list of backlog items and estimating the time for producing the artifacts for each backlog item. The purpose of a backlog item is to take a requirement description, in form of a use case description, as input and build an analysis model of our software capability as result.

\section{Definition of "Done"}
The definition of "Done" (DoD) is part of the SCRUM metrics. All members of the Scrum Team must have a shared understanding of what it means when the work is complete, to ensure transparency. \cite{scrumguide}
It is a (check-)list of items which need to be validated to consider a backlog item being “Done”. DoD is defined by the development organization to make sure that the results of multiple teams can be integrated into a releasable product. \cite{thoma1}

For our project, the result needs to match the following definition of "Done":
\begin{todolist}

\item Description of the requirement in form of a Use Case
\item Categorization of requirement (functional/non-functional, client/ server)
\item Business value of the corresponding functionality
\item Effort estimation for the implementation of the requirement
\item For UI related functions: UI prototype
\item UML Diagrams 

    $\bullet$ Use case diagram
    
    $\bullet$ Activity diagram
    
    $\bullet$ Class diagram
    
    $\bullet$ Sequence diagram
    
\item Detailed documentation (e.g. table) about who worked on the item and what has been done during the sprint.
\item Overall quality of the documentation meets general industry standards.
\item The results have been reviewed and accepted by another member of the team (tester). It needs to be documented who has performed the review.

\end{todolist}


%==============================================================================
\section{Backlog Items}

INSERT BACKLOG ITEMS LIST HERE!

\subsection{TODO}
\begin{itemize}
\item high-level-software architecture
\item pictures of UI prototypes for inserting to the ducomentation
\item talk about presentation
\end{itemize}

%==============================================================================

%==============================================================================
\section{Week One}
%==============================================================================
In the first week our team started to collect requirements for the E-Scooter rental service project. We discussed which requirements could fit and saved them in a Google Excel sheet. We also started to talk about the estimation, satisfactions and disatisfactions, the priority, what each requirement should do and when it would fit.
Then we thought about how an E-Scooter should work and interact with the customers.
In order to understand how the main concept of renting an E-Scooter is, some team members rented an E-Scooter from the company "Lime".

The task for each team member for this week was to collect more requirements for the project.
Because finding dates for further meetings turned out to be difficult, we decided to hold our meetings weekly or whenever there occur any problems which needed to be discussed, via discord. Discord is a free voice and text chat that is secure and works desktop and phone. \cite{discord}

\subsection{Division of work: week one}
\begin{table}[h]
\centering
\setlength{\tabcolsep}{12pt}
\begin{tabular}{|c|c|c|c|c|}
\hline
K. Birringer & N. Cacace & S. Hanzlik & M. Peluso & S. Stojanovic\\
\hline
\% & \% & \% & \% & \% \\ 
\hline
\end{tabular}
\end{table}

%==============================================================================
\section{Week Two}
%==============================================================================
Since we decided to use the agile SCRUM method for analyzing the E-Scooter rental service project, the team members were assigned the following roles:

\begin{itemize}
\item Scrum Master: Kendra Birringer


As Scrum Master she was resonsible for the organisation of the whole team: she organized and moderated the team meetings and wrote the protocols. 

Another task was to check and correct the spelling, grammar and contents of evertything that was written.

\item Development Team: Nader Cacace, Steffen Hanzlik, Svetozar Stojanovic

The Development Team was responsible for modeling all neccessary UML diagrams and sketching UI prototypes.

\item Tester: Marco Peluso

The Tester reviewed and accepted all results.
\end{itemize}


In the second week the team discussed about each of the collected requirements, and we decided which of them fit and which are not necessary for the software. During this discussion we gathered more requirements.

Then we started to talk about the UML diagrams and built a first use case diagram which was too big and complex and needed some adjustments. So, the task for the Development Team was to simplify the diagram and make it clearer.

Also, we built a main structure for the documentation of the project.

\subsection{Division of work: week two}
\begin{table}[h]
\centering
\setlength{\tabcolsep}{12pt}
\begin{tabular}{|c|c|c|c|c|}
\hline
K. Birringer & N. Cacace & S. Hanzlik & M. Peluso & S. Stojanovic\\
\hline
\% & \% & \% & \% & \% \\ 
\hline
\end{tabular}
\end{table}
%==============================================================================
\section{Week Three}
%===============================================================================
In week three the Development Team modified the use case diagram which was too big. They also modelled further use case diagrams which we then disussed, if they need further adjustments. At the end of this week, we finished the use case diagrams and finally added the use case documentation.

Also, after we have thought about where it could be necessary, some activity and sequence diagrams were modelled.
Regarding the sequence diagrams there were some problems.
For example, the "Check-in" diagram:

We asked ourselves how to calculate the price for a ride. After some discussion we decided to let the wallet calculate the price for the ride information from the Scooter.
First, the payment also was included in this diagram. But after reconsidering, we decided taht the payment also needs an own, more detailed diagram. And at the end, both diagrams are very connected.

Considering the acitivity diagrams, we decided to not model a diagram for "Give feedback", because it seemed too simple.

Then we started to think about the class diagram and asked ourself which classes we need and which relations the different classes could have to each other and started modelling the class diagram.

Furthermore, a first UI prototype was built with the sofware design tool "Axure". \cite{axure}

\newpage
\subsection{Division of work: week three}

\begin{table}[h]
\centering
\setlength{\tabcolsep}{12pt}
\begin{tabular}{|c|c|c|c|c|}
\hline
K. Birringer & N. Cacace & S. Hanzlik & M. Peluso & S. Stojanovic\\
\hline
\% & \% & \% & \% & \% \\ 
\hline
\end{tabular}
\end{table}

%==============================================================================
\section{Week Four}
%===============================================================================
\begin{itemize}
\item Week for adjust and modification
\item What is missing ? 
\item Adjusting a last time the Requirements 
\item on Base of the final Requirements
\item we finished the UI Prototype 
\item we finished the Class Diagrams
\item we adjusted the Activity Diagrams(no cancelation)
\item talk about the Presentation in Februrary
\item adjusted the Sequence Diagrams
\item talk about some missbuilded Diagrams and modyfied them
\item talk about Documentation in the Definition of done
\item adjust the documetation for the Use Cases 
\end{itemize}

\subsection{Division of work: week four}

\begin{table}[h]
\centering
\setlength{\tabcolsep}{12pt}
\begin{tabular}{|c|c|c|c|c|}
\hline
K. Birringer & N. Cacace & S. Hanzlik & M. Peluso & S. Stojanovic\\
\hline
\% & \% & \% & \% & \% \\ 
\hline
\end{tabular}
\end{table}
%==============================================================================
\section{Week Five}
%===============================================================================
\begin{itemize}
\item start to build the Presentation
\item discussion about which role every member get in this project and how he could integrate for this project
\item finished the activity diagram
\item finished the sequence diagram
\item finished the documentation
\item finished the presentation
\item generate Magic Draw report and add it to the Documentation
\item add the scrum protocols ot the Documentation
\item talk about the workflow table who did what and add it to the documentation
\item add References to Documentation 
\item finished the project at this moment
\item think about further work, which requirement we would build in at this moment we not implemented it
\item think about the design engeneering 
\item which functions we would build in the next time
\end{itemize}

%==============================================================================
\subsection{Division of work: week five}

\begin{table}[h]
\centering
\setlength{\tabcolsep}{12pt}
\begin{tabular}{|c|c|c|c|c|}
\hline
K. Birringer & N. Cacace & S. Hanzlik & M. Peluso & S. Stojanovic\\
\hline
\% & \% & \% & \% & \% \\ 
\hline
\end{tabular}
\end{table}
%==============================================================================

\newpage    
%==============================================================================
\begin{thebibliography}{}
\bibitem{scrumguide}
https://www.scrumguides.org/scrum-guide.html\#artifact-transparency-done 

\bibitem{thoma1}
 Prof. Dr.-Ing Peter Thoma \emph{02-3 Software Engineering Analysis (Scrum)}
 
 \bibitem{discord}
 https://discordapp.com/
 
 \bibitem{axure}
 https://www.axure.com/

\end{thebibliography}

%\listoffigures

\newpage

%==============================================================================
\section{Appendix}

\subsection{Use Case Report}
INSERT USE CASE REPORT HERE!

\subsection{UI Prototypes}
INSERT UI PROTOTYPES HERE!

\subsection{Meeting Protocols}


\end{document}
