% This is LLNCS.DEM the demonstration file of
% the LaTeX macro package from Springer-Verlag
% for Lecture Notes in Computer Science,
% version 2.2 for LaTeX2e
%
\documentclass{llncs}

%
%\usepackage{makeidx}  % allows for indexgeneration
%

\usepackage{epsfig}
%\usepackage{multirow}

\usepackage{amssymb}
\usepackage{amsmath}
\usepackage{listings}

\usepackage{color}
\definecolor{dkgreen}{rgb}{0,0.6,0}
\definecolor{gray}{rgb}{0.5,0.5,0.5}
\definecolor{mauve}{rgb}{0.58,0,0.82}

\lstset{frame=tb,
  language=Java,
  aboveskip=3mm,
  belowskip=3mm,
  showstringspaces=false,
  columns=flexible,
  basicstyle={\small\ttfamily},
  numbers=none,
  numberstyle=\tiny\color{gray},
  keywordstyle=\color{blue},
  commentstyle=\color{dkgreen},
  stringstyle=\color{mauve},
  breaklines=true,
  breakatwhitespace=true,
  tabsize=3
}

\begin{document}

\makeatletter
\renewcommand\subsubsection{\@startsection{subsubsection}{2}{\z@}%
                       {-18\p@ \@plus -4\p@ \@minus -4\p@}%
                       {\lineskip}%
                       {\normalfont\normalsize\bfseries\boldmath}}
\makeatother


\title{Software Engeneering Analysis Sientific Report}

 % abbreviated author list (for running head)
\author{Kendra Birringer (1229372) \\ Nader Cacace (1208115) \\ Steffen Hanzlik (1207417) \\ Marco Peluso (1228849) \\ Svetozar Stojanovic (1262287)}
%%% modified list of authors for the TOC (add the affiliations)

\institute{
Frankfurt University of Applied Sciences \\
}

\maketitle              % typeset the title of the contribution
%=============================================================================
\section{Exercise 2}
%=============================================================================
%=============================================================================
\subsection{Fahrzeug }
%=============================================================================

\begin{lstlisting}[basicstyle=\footnotesize\ttfamily, numbers=left, stepnumber=1, numberstyle = \normalsize]
void Fahrzeug::setName(const char *n)
{
     if (name != nullptr) {
     	delete name;
     	name = nullptr; // handling dangling pointer to freed memory
     }

     if (n != nullptr) { // handling empty pointer as parameter, preventing strlen from crashing
     	name = new char[strlen(n)+1];
     	strcpy(name, n);
     }
}



Fahrzeug& operator=(const Fahrzeug &other) {
	
}

\end{lstlisting}
\footnotesize{Figure 1 Fahrzeug Program}\newline


%%%%%%%%%%%%%%%%%%%%%%%%%%%%%%%%%%%%%%%%%%%%%%%%%%%%%%%%%%%%%%%%%%%%%%%%%%%%%%%
%=============================================================================
\section{Exercise 3}
%=============================================================================
%=============================================================================
\subsection{Ebook Headerfile}
%=============================================================================

\begin{lstlisting}[basicstyle=\footnotesize\ttfamily, numbers=left, stepnumber=1, numberstyle = \normalsize]
#ifndef _EBOOK_H_
#define _EBOOK_H_
#include <string>
#include <iostream>
using namespace std;
class EBook
{
private:
	string title, content; 
public:
	EBook() :title{ "" }, content{ "" } {};
	EBook(string title, string content) :title{ title }, content{ content }{};
	void SetTitle(string title);
	string GetTitle()const;
	void SetContent(string content);
	string GetContent()const;
	void print();
	friend ostream &operator<<(ostream &output, const EBook& book);
};


#endif // !_EBOOK_H_

\end{lstlisting}
\footnotesize{Figure 2 Header of Ebook Program}\newline

\newpage
%=============================================================================
\subsection{Ebook Class}
%=============================================================================
\begin{lstlisting}[basicstyle=\footnotesize\ttfamily, numbers=left, stepnumber=1, numberstyle = \normalsize]
#include "eBook.h"
#include <iostream>
using namespace std;
void EBook::SetTitle(string title)
{
	if (title!="")
	{
		this->title = title;
	}
	else
	{
		cout << "Title not set!" << endl;
	}
}

string EBook::GetTitle() const
{
	return this->title;
}

void EBook::SetContent(string content)
{
	if (content != "")
	{
		this->content = content;
	}
	else
	{
		cout << "Content not set!" << endl;
	}
}

string EBook::GetContent() const
{
	return this->content;
}

void EBook::print()
{
	cout << "Title: " << this->title << '\n';
	cout << "Content: " << '\n' <<this->content << '\n';

}

ostream & operator<<(ostream & output, const EBook & book)
{
	output << "Title: " << book.title << '\n' << "Content: " << book.content << '\n';
	//or alternatively 

	//output<<book.print()<<'\n';
	return output;
}
\end{lstlisting}
\footnotesize{Figure 3 Ebook class}\newline
%=============================================================================
\subsection{Main Class}
%=============================================================================

\begin{lstlisting}
#include <iostream>
#include "eBook.h"
int main() {
	EBook book("Brown Fox", "The quick brown fox jumps over the lazy dog.");
	std::cout << book; 


	return 0; 

\end{lstlisting}
\footnotesize{Figure 4 Main class}
%=============================================================================
\section{Exercise 4}
%=============================================================================
%=============================================================================
\subsection{Box Headerfile}
%=============================================================================

\begin{lstlisting}[basicstyle=\footnotesize\ttfamily, numbers=left, stepnumber=1, numberstyle = \normalsize]
#ifndef _BOX_H_
#define _BOX_H_
class Box
{
private:
	double xMin, xMax, yMin, yMax;
public:
	
	Box():xMin{ 0.0 }, xMax{ 0.0 }, yMin{ 0.0 }, yMax{ 0.0 }{}
	double GetXMin() const { return xMin; }
	double GetXMax() const { return xMax; }
	double GetYMin() const { return yMin; }
	double GetYMax() const { return yMax; }
	void SetXMax(double val);
	void SetXMin(double val);
	void SetYMin(double val);
	void SetYMax(double val);
	friend Box operator+(Box left, Box right);
	void print(); 
};

#endif // !_BOX_H_
\end{lstlisting}
\footnotesize{Figure 5 Header of Box}
%=============================================================================
\subsection{Circle Headerfile}
%=============================================================================
\begin{lstlisting}[basicstyle=\footnotesize\ttfamily, numbers=left, stepnumber=1, numberstyle = \normalsize]
#ifndef _CIRCLE_H_
#define _CIRLE_H_
#include "Form.h"
class Circle: public Form
{
private: 
	double radius; 
public:
	Circle() 
	{
		Form();
		this->box.SetXMax(0.0);
		this->box.SetXMin(0.0);
		this->box.SetYMax(0.0);
		this->box.SetYMin(0.0);
	}
	Circle(double rad):radius{rad}{}

	//MOVE FOR CIRCLE
	void Move(double dX, double dY);
	void SetUpBox();
private:
	void MoveBox(double dX = 0, double dY = 0);
	
};

#endif // !_CIRCLE_H_
\end{lstlisting}
\footnotesize{Figure 6 Header of Circle}
%=============================================================================
\subsection{Form Headerfile}
%=============================================================================
\begin{lstlisting}[basicstyle=\footnotesize\ttfamily, numbers=left, stepnumber=1, numberstyle = \normalsize]
#ifndef _FORM_H_
#define _FORM_H_
#include "Box.h"
class Form
{
private:
	double xCenter, yCenter;  
protected:	
	Box box; 
	
public:
	Form()
	{
		this->xCenter = 0.0;
		this->yCenter = 0.0;
	}
	
	void Move(double dX, double dY) {
		this->xCenter += dX;
		this->yCenter += dY;
		
	}
	Box & GetBoxRef() { //how to get const ref???
		return box; 
	}
};

#endif // !_FORM_H_
\end{lstlisting}
\footnotesize{Figure 7 Header of Form}
%=============================================================================
\subsection{Rectangle Headerfile}
%=============================================================================
\begin{lstlisting}[basicstyle=\footnotesize\ttfamily, numbers=left, stepnumber=1, numberstyle = \normalsize]
#ifndef _RECTANGLE_H_
#define _RECTANGLE_H_
#include "Form.h"
class Rectangle: public Form
{
private:
	double width, height; 

public:
	Rectangle() 
	{
		Form();
		this->box.SetXMax(0.0);
		this->box.SetXMin(0.0);
		this->box.SetYMax(0.0);
		this->box.SetYMin(0.0);

	}
	Rectangle(double h, double w):height{h},width{w}{}
	//MOVE FOR RECT
	void Move(double dX, double dY); 
	void SetUpBox();
private:
	void MoveBox(double dX = 0, double dY = 0);
	
};



#endif // !_RECTANGLE_H_
\end{lstlisting}
\footnotesize{Figure 8 Header of Rectangle}
%=============================================================================
\subsection{Box Class}
%=============================================================================

\begin{lstlisting}[basicstyle=\footnotesize\ttfamily, numbers=left, stepnumber=1, numberstyle = \normalsize]
#include "Box.h"
#include <iostream>
#include <algorithm>
using namespace std; 

void Box::SetXMax(double val)
{
	this->xMax = val; 

}

void Box::SetXMin(double val)
{
	this->xMin = val; 

}

void Box::SetYMin(double val)
{
	this->yMin = val;
}

void Box::SetYMax(double val)
{
	this->yMax = val; 
}

void Box::print()
{
	cout << "xMax: " << xMax << endl; 
	cout << "xMin: " << xMin << endl;
	cout << "yMax: " << yMax << endl;
	cout << "yMin: " << yMin << endl;

}

Box operator+(Box left, Box right)
{
	Box newLeft, newRight;
	if (left.GetXMax()>right.GetXMax())
	{
		newLeft = right;
		newRight = left;
	}
	else
	{
		newLeft = left;
		newRight = right;
	}
	Box result; 
	if (right.GetXMin() < left.GetXMax() && right.GetYMin() < left.GetYMax()) //check if the boxes collide 
	{
		
		result.SetXMin(min(newLeft.GetXMin(), newRight.GetXMin()));
		result.SetXMax(max(newLeft.GetXMax(), newRight.GetXMax()));
		result.SetYMin(min(newLeft.GetYMin(), newRight.GetYMin()));
		result.SetYMax(max(newLeft.GetYMax(), newRight.GetYMax()));
		return result;
	}
	else
	{
		cout << "The boxes of these two objects don't collide." << '\n';
	}
	 
}
\end{lstlisting}
\footnotesize{Figure 9 Box Class}
%=============================================================================
\subsection{Circle Class}
%=============================================================================
\begin{lstlisting}[basicstyle=\footnotesize\ttfamily, numbers=left, stepnumber=1, numberstyle = \normalsize]
#include "Circle.h"

void Circle::SetUpBox()
{
	this->box.SetXMax(this->radius);

	this->box.SetXMin(-this->radius);

	this->box.SetYMax(this->radius);

	this->box.SetYMin(-this->radius);
}

void Circle::Move(double dX, double dY)
{
	Form::Move(dX, dY);
	MoveBox(dX, dY);
}

void Circle::MoveBox(double dX, double dY)
{
	this->box.SetXMax(box.GetXMax() + dX);

	this->box.SetXMin(box.GetXMin() + dX);

	this->box.SetYMax(box.GetYMax() + dY);

	this->box.SetYMin(box.GetYMin() + dY);
}

\end{lstlisting}
\footnotesize{Figure 10 Circle Class}

%=============================================================================
\subsection{Rectangle Class}
%=============================================================================
\begin{lstlisting}[basicstyle=\footnotesize\ttfamily, numbers=left, stepnumber=1, numberstyle = \normalsize]
#include "Rectangle.h"

void Rectangle::Move(double dX, double dY)
{
	Form::Move(dX, dY);
	MoveBox(dX, dY);
}

void Rectangle::MoveBox(double dX, double dY)
{
	this->box.SetXMax(box.GetXMax()+dX);

	this->box.SetXMin(box.GetXMin()+dX);

	this->box.SetYMax(box.GetYMax()+dY);

	this->box.SetYMin(box.GetYMin()+dY);
}

void Rectangle::SetUpBox()
{

	this->box.SetXMax(width / 2);

	this->box.SetXMin(-width / 2);

	this->box.SetYMax(height / 2);

	this->box.SetYMin(-height / 2);
}
\end{lstlisting}
\footnotesize{Figure 11 Rectangle Class}
%=============================================================================
\subsection{Main Class}
%=============================================================================
\begin{lstlisting}[basicstyle=\footnotesize\ttfamily, numbers=left, stepnumber=1, numberstyle = \normalsize]
#include <iostream>
#include <string>
#include "Circle.h"
#include "Rectangle.h"
using namespace std; 
bool InputIsCircle(string); //checks if the user typed 'circle', returns bool
bool InputIsRect(string);	//checks if the user typed 'rectangle', returns bool
Circle* CircleCreator(bool isTrue); //asks for needed values and calls circle constructor
Rectangle* RectCreator(bool isTrue);//asks for needed values and calls rectangle constructor
Box AddBoxes(Circle* c1, Circle* c2, Rectangle* r1, Rectangle* r2); 
int main() {

	double movX, movY;	//arguments for Move(...) function
	
	string prompt="";
	
	std::cout << "_______________________________________" << endl; 

	Circle *circle1=NULL;
	Rectangle *rect1=NULL;

	cout << "Enter first form (rectangle or circle): ";
	cin >> prompt;
	
	if (InputIsCircle(prompt))
	{
		circle1 = CircleCreator(InputIsCircle(prompt));
		circle1->GetBoxRef().print();

		cout << "Move Circle in X direction for: ";
		cin >> movX;
		cout << "Move Circle in Y direction for: ";
		cin >> movY;
		
		circle1->Move(movX, movY);
		cout << "After Move is called: " << endl;
		circle1->GetBoxRef().print();

	}
	else if (InputIsRect(prompt))
	{
		rect1 = RectCreator(InputIsRect(prompt));
		rect1->GetBoxRef().print();

		cout << "Move Rectangle in X direction for: ";
		cin >> movX;
		cout << "Move Rectangle in Y direction for: ";
		cin >> movY;
		
		rect1->Move(movX, movY);
		cout << "After Move is called: " << endl;
		rect1->GetBoxRef().print();
	}
	
	Circle *circle2 = NULL;
	Rectangle *rect2 = NULL;
	cout << "Enter second form (rectangle or circle): ";
	cin >> prompt;
	if (InputIsCircle(prompt))
	{
		circle2 = CircleCreator(InputIsCircle(prompt));
		circle2->GetBoxRef().print();

		cout << "Move Circle in X direction for: ";
		cin >> movX;
		cout << "Move Circle in Y direction for: ";
		cin >> movY;

		circle2->Move(movX, movY);
		cout << "After Move is called: " << endl;
		circle2->GetBoxRef().print();

	}
	else if (InputIsRect(prompt))
	{
		rect2 = RectCreator(InputIsRect(prompt));
		rect2->GetBoxRef().print();

		cout << "Move Rectangle in X direction for: ";
		cin >> movX;
		cout << "Move Rectangle in Y direction for: ";
		cin >> movY;

		rect2->Move(movX, movY);
		cout << "After Move is called: " << endl;
		rect2->GetBoxRef().print();

	}
	
	

	//ADD BOUNDING BOXES AND PRODUCE NEW ONE AS SUM
	
	
	
	Box boundingBox;

	cout << "Bounding Box: " << endl; 
	boundingBox = AddBoxes(circle1, circle2, rect1, rect2);
	if (!(boundingBox.GetXMax()==0.0 && boundingBox.GetXMin()==0.0 && boundingBox.GetYMin()==0.0 && boundingBox.GetYMax()==0.0))
	{
		boundingBox.print();
	}
	
	cout << "_______________________________________" << endl;

	delete circle1, rect1, circle2, rect2; 

	return 0; 
}
Box AddBoxes(Circle* c1, Circle* c2, Rectangle* r1, Rectangle* r2) {
	
	Box result;
	if (c1==NULL && c2==NULL)
	{
		result = r1->GetBoxRef() + r2->GetBoxRef();
		return result;
	}
	else if (c1==NULL && r2==NULL)
	{
		result = r1->GetBoxRef()+ c2->GetBoxRef();
		return result;
	}
	else if (r1==NULL && c2==NULL)
	{
		result = c1->GetBoxRef()+ r2->GetBoxRef();
		return result;
	}
	else if (r1==NULL && r2==NULL)
	{
		result = c1->GetBoxRef() + c2->GetBoxRef();
		return result;
	}
}
bool InputIsCircle(string prompt)  {
	string circle = "circle";
	bool result = false; 
	if (prompt.compare(circle) == 0)
	{
		result = true; 
		
	}
	else
	{
		return result; 
	}
}
bool InputIsRect(string prompt) {
	string rect = "rectangle";
	bool result = false;
	if (prompt.compare(rect) == 0)
	{
		result = true; 
		
	}
	else
	{
		return result;
	}
}
Circle* CircleCreator(bool isTrue) {
	if (isTrue)
	{
		double rad;
		
		cout << "Enter radius: ";
		cin >> rad;
		Circle *circle=new Circle(rad);
		circle->SetUpBox();
		return circle;
	}
	else
	{
		return NULL; 
	}

}
Rectangle* RectCreator(bool isTrue) {
	if (isTrue)
	{
		double h, w;
		cout << "Enter height: ";
		cin >> h;
		cout << "Enter width: ";
		cin >> w;
		Rectangle *rect=new Rectangle(h, w);
		rect->SetUpBox();
		return rect; 
	}
	else
	{
		return NULL;
	}
}
\end{lstlisting}
\footnotesize{Figure 12 Main Class}

%=============================================================================
\section{Foundations}
%=============================================================================



%-----------------------------------------------------------------------------
%\subsection{}
%-----------------------------------------------------------------------------



%-----------------------------------------------------------------------------
%\subsection{}
%-----------------------------------------------------------------------------

%-----------------------------------------------------------------------------
%\subsection{}
%-----------------------------------------------------------------------------


%=============================================================================
%\section{}
%=============================================================================



%-----------------------------------------------------------------------------
%\subsection{}
%-----------------------------------------------------------------------------


%=============================================================================
%\section{}
%=============================================================================



%=============================================================================
\section{Conclusion}
%=============================================================================


\end{document}

\newpage

